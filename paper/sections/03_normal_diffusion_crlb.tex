% ============================================================
% 03_normal_diffusion_crlb.tex
% ============================================================

\section{Normal Diffusion: Fisher Information and Cramér--Rao Bounds}
\label{sec:normal_diffusion_crlb}

We begin with the simplest and analytically tractable case: free normal diffusion
with a Gaussian propagator. This section establishes Fisher information and
Cramér--Rao lower bounds (CRLB) for time inference, including explicit numerical
coefficients and confidence levels. The results serve as a reference baseline for
all subsequent generalizations.

\subsection{Model Definition}

Consider a $d$-dimensional diffusive process with diffusion coefficient $D>0$.
For normal diffusion, each spatial coordinate has variance $2Dt$.
The latent spatial distribution at time $t$ is therefore Gaussian:
\begin{equation}
p(x \mid t) = \frac{1}{(4\pi D t)^{d/2}}
\exp\!\left( - \frac{\|x\|^2}{4 D t} \right), \quad x \in \mathbb{R}^d .
\end{equation}

In the presence of a Gaussian observation noise floor $\sigma_0^2$ (PSF or localization
error), the observed distribution becomes
\begin{equation}
p_{\mathrm{obs}}(x \mid t)
= \mathcal{N}\!\left( 0, \, s(t) I_d \right),
\qquad
s(t) := 2 D t + \sigma_0^2 .
\end{equation}

We first treat the case where $D$ and $\sigma_0$ are known parameters; joint estimation
and the associated information loss are addressed separately.

\subsection{Score Function}

The log-likelihood for a single observation is
\begin{equation}
\log p_{\mathrm{obs}}(x \mid t)
= -\frac{d}{2}\log(2\pi s(t))
  -\frac{\|x\|^2}{2 s(t)} .
\end{equation}

Differentiating with respect to $t$ yields the score function
\begin{equation}
\frac{\partial}{\partial t}\log p_{\mathrm{obs}}(x \mid t)
= D\left(
    \frac{\|x\|^2}{s(t)^2}
    - \frac{d}{s(t)}
  \right),
\end{equation}
which is unbiased and has zero mean under $p_{\mathrm{obs}}(x\mid t)$.

\subsection{Fisher Information}

The Fisher information for a single observation is defined as
\begin{equation}
I_1(t)
:= \mathbb{E}\!\left[
    \left(
      \frac{\partial}{\partial t}\log p_{\mathrm{obs}}(X \mid t)
    \right)^2
  \right].
\end{equation}

Using the Gaussian moments
\[
\mathbb{E}[\|X\|^2] = d\, s(t),
\qquad
\mathbb{E}[\|X\|^4] = d(d+2)\, s(t)^2 ,
\]
a direct calculation gives
\begin{equation}
I_1(t)
= \frac{2 d D^2}{s(t)^2}
= \frac{2 d D^2}{\left(2 D t + \sigma_0^2\right)^2}.
\end{equation}

For $N_{\mathrm{eff}}$ statistically independent (or effectively independent)
observations, Fisher information is additive:
\begin{equation}
I_{\mathrm{eff}}(t)
= N_{\mathrm{eff}} I_1(t).
\end{equation}

\subsection{Cramér--Rao Lower Bound}

For any unbiased estimator $\hat t$ of the time parameter,
\begin{equation}
\mathrm{Var}(\hat t)
\ge \frac{1}{I_{\mathrm{eff}}(t)}
= \frac{\left(2 D t + \sigma_0^2\right)^2}
       {2 d D^2 N_{\mathrm{eff}}}.
\end{equation}

Introducing a confidence multiplier $z$ (e.g., $z=1$ for one standard deviation,
$z=1.96$ for 95\% confidence), we define the minimal distinguishable time increment
\begin{equation}
\Delta t_{\min}^{\mathrm{CRLB}}(t; z)
:= z \sqrt{\mathrm{Var}(\hat t)}
= z \frac{2 D t + \sigma_0^2}{\sqrt{2 d}\, D \sqrt{N_{\mathrm{eff}}}} .
\end{equation}

The factor $z$ fixes a confidence convention and does not affect the asymptotic
scaling behavior of the bound.

\subsection{Limiting Regimes}

Two experimentally relevant regimes follow immediately.

\paragraph{PSF-dominated regime ($2Dt \ll \sigma_0^2$).}
\begin{equation}
\Delta t_{\min}^{\mathrm{CRLB}}
\approx
z \frac{\sigma_0^2}{\sqrt{2 d}\, D \sqrt{N_{\mathrm{eff}}}},
\end{equation}
which is independent of $t$.

\paragraph{Diffusion-dominated regime ($2Dt \gg \sigma_0^2$).}
\begin{equation}
\Delta t_{\min}^{\mathrm{CRLB}}
\approx
z \sqrt{\frac{2}{d}}
\frac{t}{\sqrt{N_{\mathrm{eff}}}} .
\end{equation}

These expressions provide exact bounds within the stated Gaussian diffusion model
and observation assumptions.

\subsection{Interpretation}

The bounds above quantify a fundamental limitation on inferring elapsed time from
spatial data. They do not imply any modification of the underlying dynamics and do
not introduce a discrete time scale. Instead, they express the degradation of temporal
information caused by finite spatial resolution, finite statistics, and noise. The
resulting limitation is purely operational and persists independently of the
specific estimator employed.
