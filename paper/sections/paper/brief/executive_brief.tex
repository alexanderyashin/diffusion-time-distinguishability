\documentclass[11pt,a4paper]{article}

\usepackage{amsmath,amssymb}
\usepackage{geometry}
\usepackage{microtype}
\usepackage{hyperref}

\geometry{margin=2.2cm}

\title{\textbf{Operational Limits of Time Distinguishability\\
in Diffusive Dynamics}}
\author{Alexander Yashin\\
\small Independent Researcher, Leipzig/Halle, Germany}
\date{}

\begin{document}
\maketitle

\vspace{-1em}

\section*{Scope of this note}

This document is a condensed operational summary of a full theoretical
and numerical study.
All derivations, simulations, and reproducibility details
are provided in the extended version available via Zenodo/GitHub.

\section*{Problem}

In stochastic systems such as diffusion, time is not directly observable.
It is inferred from spatial statistics.
This raises a precise operational question:

\begin{quote}
\emph{What is the minimal time difference $\Delta t$ that can be
distinguished from data, given finite statistics and measurement noise?}
\end{quote}

This is not a question of dynamics,
but of \emph{information and inference} \cite{Fisher1925,Cramer1946,Kay1993}.

\section*{Framework}

Let $P(x|t)$ denote the propagator of a diffusive process.
Time $t$ is treated as an \emph{estimable parameter}.

Distinguishability between $t$ and $t+\Delta t$ is quantified using:
\begin{itemize}
\item Fisher information $I(t)$ and the Cram\'er--Rao bound,
\item Kullback--Leibler divergence for hypothesis testing \cite{Kullback1951}.
\end{itemize}

For normal diffusion in $d$ dimensions:
\[
P(x|t)=\frac{1}{(4\pi Dt)^{d/2}}
\exp\!\left(-\frac{x^2}{4Dt}\right),
\qquad
\langle x^2\rangle = 2dDt,
\]
following the classical theory of Brownian motion \cite{Einstein1905,Risken1996}.

\section*{Key Result I: Statistical Limit}

The Fisher information for estimating $t$ from $N$ independent samples is:
\[
I_N(t)=\frac{Nd}{2t^2}.
\]

Thus,
\[
\mathrm{Var}(\hat t)\ge \frac{2t^2}{Nd},
\qquad
\frac{\Delta t_{\min}}{t}\sim \frac{1}{\sqrt{Nd}}.
\]

\textbf{Interpretation:}
even with perfect spatial resolution,
temporal resolution is limited by finite information.

\section*{Key Result II: Photon-Limited Regime}

In optical measurements, variance estimation is limited by Poisson
photon statistics.
Let $\Phi$ denote photon flux.

Under Gaussian point-spread functions and shot-noise-limited detection,
self-consistent uncertainty propagation yields:
\[
\boxed{
\Delta t_{\min}\;\propto\;\Phi^{-1/3}
}
\]

This scaling:
\begin{itemize}
\item is nontrivial,
\item does not follow from dimensional analysis,
\item arises from the interplay of inference and noise,
\end{itemize}
and is consistent with precision limits in single-particle microscopy
\cite{Ober2004,Balzarotti2017}.

\section*{Key Result III: Anomalous Diffusion (Interpretive)}

For anomalous diffusion
$\langle x^2\rangle=2dD_\alpha t^\alpha$,
a minimal resolution-based lower bound is:
\[
\Delta t_{\mathrm{MSD}}(t)\sim
\frac{\Delta x^2}{2d\alpha D_\alpha}\,t^{1-\alpha}.
\]

This criterion is \emph{interpretive} and serves as intuition.
Information-theoretic bounds in the photon-limited regime yield:
\[
\Delta t_{\min}(t)\propto t^{(2-\alpha)/3},
\]
in agreement with general anomalous diffusion theory
\cite{Metzler2000,Sokolov2012}.

\begin{itemize}
\item Subdiffusion ($\alpha<1$): temporal resolution degrades with time.
\item Superdiffusion ($\alpha>1$): temporal resolution improves.
\end{itemize}

These qualitative distinctions are experimentally testable.

\section*{Falsifiability}

The framework is falsified for a given regime if systematic and
statistically significant deviations are observed from the predicted
scaling laws, including:
\begin{itemize}
\item $\Delta t_{\min}\propto t/\sqrt{N}$ in the statistical regime,
\item $\Delta t_{\min}\propto \Phi^{-1/3}$ in photon-limited experiments,
\item consistency between anomalous exponent $\alpha$ and temporal scaling.
\end{itemize}

Numerical simulations reproducing all predicted scalings
are provided in the accompanying repository.

\section*{What This Is (and Is Not)}

\begin{itemize}
\item This is \textbf{not} a quantization of time.
\item This \textbf{is} an operational limit of temporal inference.
\item No metaphysical assumptions are required.
\end{itemize}

Time remains continuous in dynamics;
discreteness emerges only at the level of distinguishability,
in the operational sense advocated by Bridgman \cite{Bridgman1927}.

\section*{Status}

This result is:
\begin{itemize}
\item mathematically explicit,
\item information-theoretic,
\item experimentally falsifiable,
\item conditional only on stated assumptions.
\end{itemize}

A full derivation, numerical simulations, and reproducibility notes
are available in the extended version (Zenodo / GitHub).

\bibliographystyle{unsrt}
\bibliography{references}

\end{document}
