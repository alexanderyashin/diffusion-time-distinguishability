% ============================================================
% 01_assumptions_scope.tex
% ============================================================

\section{Assumptions, Scope, and Model Classes}
\label{sec:assumptions_scope}

This work derives operational limits on \emph{time distinguishability} from spatial
data generated by diffusive processes. To avoid hidden assumptions and ensure
falsifiability, we explicitly enumerate all assumptions, define the scope of validity,
and classify the model families considered. All subsequent theorems, bounds, and
scalings are conditional on this section.

\subsection{Operational Goal}

The central goal is not to postulate time as a primitive parameter, but to determine
the minimal increment of time that can be statistically distinguished from zero using
spatial observations under finite resources. Time is treated as an inferred parameter,
and its distinguishability is defined operationally as a property of probability
distributions induced by the dynamics, quantified via estimation and hypothesis
testing theory.

\subsection{Observation Paradigm}

We consider \emph{snapshot-based} observations of a spatial distribution generated
by an underlying stochastic process. Each observation corresponds to one of the
following experimentally relevant scenarios:

\begin{itemize}
  \item A single spatial snapshot (e.g., fluorescence intensity profile) generated
        after an unknown elapsed time.
  \item A finite set of particle positions sampled from the spatial distribution at
        an unknown or uncertain time.
  \item Two closely spaced snapshots with uncertain inter-frame delay.
\end{itemize}

Time is inferred solely from spatial statistics of the observed data; no direct time
stamps or dynamical trajectories are assumed unless explicitly stated.

\subsection{Assumptions}

We group assumptions into structural, statistical, and observational categories.

\paragraph{A1 (Free stochastic evolution).}
The latent dynamics are stochastic and Markovian at the level of the propagator.
For normal diffusion this corresponds to the diffusion equation; for anomalous
diffusion, specified generalizations apply.

\paragraph{A2 (Snapshot statistics).}
Observations correspond to instantaneous snapshots of the spatial distribution.
Temporal averaging during exposure is neglected at the theoretical level and treated
as an effective correlation or blur effect when relevant. Corrections of this type are
absorbed into the effective sample size defined below.

\paragraph{A3 (Gaussian observation noise).}
Observed positions include additive, zero-mean Gaussian noise with variance
$\sigma_0^2$ per spatial dimension, representing the point-spread function (PSF) or
localization error:
\[
x_{\mathrm{obs}} = x + \varepsilon, \quad \varepsilon \sim \mathcal{N}(0,\sigma_0^2).
\]

\paragraph{A4 (Finite statistics).}
Only a finite number $N$ of samples (or an effective number $N_{\mathrm{eff}}$ in the
presence of correlations) is available for inference.

\paragraph{A5 (Photon-limited acquisition).}
In optical measurements, the total number of detected photons $N_\gamma$ is finite
and proportional to the exposure duration:
\[
N_\gamma = \Phi \,\Delta t,
\]
where $\Phi$ is the photon flux and $\Delta t$ denotes a controllable acquisition or
exposure interval. This $\Delta t$ should be distinguished from the inferred minimal
distinguishable time increment $\Delta t_{\min}$ introduced later. The analysis
explicitly accounts for the self-consistent coupling between exposure time and
information acquisition.

\paragraph{A6 (Regularity conditions).}
Probability distributions are assumed sufficiently smooth in the time parameter to
permit Fisher information and local KL expansions. Explicit regularity conditions are
stated where they are required.

\subsection{Effective Sample Size and Correlations}

When observations are correlated (e.g., due to motion blur, finite exposure, or
trajectory-based sampling), the number of statistically independent samples is
reduced. We define an effective sample size
\[
N_{\mathrm{eff}} = \frac{N}{1 + 2\sum_{k=1}^{\infty} \rho_k},
\]
where $\rho_k$ are autocorrelations of the sufficient statistic used for inference.
This expression is valid under standard assumptions of weak stationarity and
summable correlations, such that a central limit theorem applies. All asymptotic
bounds in this work hold with the substitution $N \to N_{\mathrm{eff}}$ when these
conditions are satisfied.

\subsection{Model Classes}

The analysis is organized by model class.

\paragraph{Class I: Normal diffusion (Gaussian propagator).}
The spatial distribution at time $t$ is Gaussian with variance proportional to $t$.
This class admits closed-form Fisher information and exact Cramér--Rao bounds.

\paragraph{Class II: Gaussian anomalous diffusion.}
Processes with Gaussian propagators but anomalous mean-square displacement,
$\langle x^2\rangle \propto t^\alpha$ (e.g., fractional Brownian motion or scaled
Brownian motion). Fisher information exists but exhibits nontrivial time scaling
determined by the anomalous exponent $\alpha$.

\paragraph{Class III: Non-Gaussian anomalous diffusion.}
Processes with non-Gaussian propagators (e.g., continuous-time random walks or
Lévy flights). Universal analytic Fisher information bounds generally do not exist
due to the absence of regular likelihood structure. In this class, time
distinguishability is defined operationally via data-driven Kullback--Leibler
divergence and hypothesis testing between empirical distributions.

\subsection{Scope and Limitations}

This work does \emph{not} assume:
\begin{itemize}
  \item The existence of a fundamental or quantized time step.
  \item Infinite spatial or temporal resolution.
  \item Knowledge of all model parameters unless explicitly stated.
\end{itemize}

The results provide lower bounds on time distinguishability under the stated
assumptions. Violation of the predicted scalings under controlled conditions
constitutes direct falsification. Explicit no-go statements and numerical validation
are provided in Appendices~C and~D.

\subsection{Roadmap}

Subsequent sections proceed as follows. Section~\ref{sec:problem_setup} formalizes
time distinguishability as an inference problem. Sections on Fisher information and
KL divergence derive exact and asymptotic bounds for normal diffusion, followed by
extensions to photon-limited and anomalous regimes. Numerical verification and
experimental protocols are discussed separately.
