% ============================================================
% Appendix D
% Numerical Validation, Figures, and Joint Estimation
% (FULL REPLACEMENT — content preserved, updated for strictness)
% ============================================================

\appendix
\section{Appendix D: Numerical Validation, Figures, and Joint Estimation}
\label{app:numerical_validation}

\subsection{D.1 Scope and purpose of numerical validation}

This appendix provides quantitative numerical validation of the theoretical limits derived in the main text and Appendices A–C.
The goals are fourfold:

\begin{enumerate}
    \item To verify the Fisher-information-based Cramér–Rao lower bounds (CRLB) via Monte Carlo simulations.
    \item To confirm the photon-limited temporal resolution scaling $\Delta t_{\min} \propto \Phi^{-1/3}$ numerically.
    \item To demonstrate the practical equivalence between estimation-theoretic (CRLB) and hypothesis-testing (KL divergence) formulations.
    \item To assess the robustness of the time distinguishability limit under joint estimation of $(t, D, \sigma_0)$.
\end{enumerate}

All simulations are fully reproducible and generated from the accompanying source code (see Appendix~B).
Throughout, $N_{\mathrm{eff}}$ denotes the effective number of statistically independent samples, and all plotted quantities are generated by the scripts that produce the corresponding figure files in \texttt{figs/}.

---

\subsection{D.2 Monte Carlo verification of the CRLB for normal diffusion}

We consider free Brownian motion in $d$ dimensions with propagator
\[
P(\mathbf{x}|t) = (4\pi D t)^{-d/2} \exp\!\left(-\frac{|\mathbf{x}|^2}{4Dt}\right),
\]
with known diffusion coefficient $D$ and negligible initial width $\sigma_0 = 0$.

For each Monte Carlo trial, $N_{\mathrm{eff}}$ independent samples are drawn and the maximum-likelihood estimator $\hat{t}$ is computed.
The empirical variance $\mathrm{Var}(\hat{t})$ is compared to the theoretical bound
\[
\mathrm{Var}(\hat{t}) \ge \frac{2 t^2}{d N_{\mathrm{eff}}}.
\]

\begin{figure}[h!]
\centering
\includegraphics[width=0.75\textwidth]{../figs/fig_D1_crlb_vs_Neff.pdf}
\caption{Monte Carlo verification of the CRLB for time estimation in normal diffusion.
Points show empirical variance of $\hat{t}$; solid line shows the CRLB.
Agreement is asymptotically exact as $N_{\mathrm{eff}} \to \infty$.}
\label{fig:D1}
\end{figure}

The simulations confirm that no unbiased estimator can asymptotically outperform the CRLB, establishing the bound as a fundamental information-theoretic limit rather than an artifact of a particular estimation procedure.

---

\subsection{D.3 Photon-limited regime and $\Phi^{-1/3}$ scaling}

We now consider the experimentally relevant photon-limited regime.
Let the observed width satisfy
\[
\sigma_{\mathrm{obs}}^2(t) = \sigma_0^2 + 2 d D t,
\]
and assume Poisson photon statistics with flux $\Phi$ photons per unit time.

For Gaussian profile fitting under Poisson noise, the variance of an unbiased estimator of $\sigma^2$ obeys
\[
\mathrm{Var}(\widehat{\sigma^2}) \simeq \kappa \frac{\sigma_{\mathrm{obs}}^4}{N_\gamma},
\quad N_\gamma = \Phi \Delta t,
\]
with $\kappa \approx 2$ for maximum-likelihood estimation \cite{Ober2004}.

Imposing self-consistency of time estimation yields
\[
\Delta t_{\min} \sim
\left(\frac{\sigma_{\mathrm{obs}}^4}{2 d^2 D^2 \Phi}\right)^{1/3}.
\]

\begin{figure}[h!]
\centering
\includegraphics[width=0.75\textwidth]{../figs/fig_D2_photon_scaling.pdf}
\caption{Photon-limited temporal resolution.
Log–log plot of $\Delta t_{\min}$ versus photon flux $\Phi$.
The slope $-1/3$ is confirmed numerically, validating the self-consistent scaling prediction.}
\label{fig:D2}
\end{figure}

This scaling is fundamentally distinct from the familiar $\sim \Phi^{-1/2}$ behavior of spatial localization precision, reflecting the indirect and self-referential nature of temporal inference through diffusive spreading.

---

\subsection{D.4 CRLB versus KL-divergence distinguishability}

To connect estimation and hypothesis testing, we numerically evaluate the Kullback–Leibler divergence
\[
D_{\mathrm{KL}}\!\left(P(\cdot|t) \Vert P(\cdot|t+\Delta t)\right)
\]
from Monte Carlo samples and compare it with the quadratic Fisher approximation
\[
D_{\mathrm{KL}} \approx \tfrac{1}{2} I(t) (\Delta t)^2.
\]

\begin{figure}[h!]
\centering
\includegraphics[width=0.75\textwidth]{../figs/fig_D3_kl_vs_crlb.pdf}
\caption{Empirical equivalence between CRLB and KL distinguishability.
Points: Monte Carlo KL divergence.
Line: Fisher quadratic approximation.
The equivalence holds in the small-$\Delta t$ regime relevant for time resolution limits.}
\label{fig:D3}
\end{figure}

This demonstrates that the temporal distinguishability limit is formulation-independent: both optimal estimation and optimal hypothesis testing lead to identical operational bounds.

---

\subsection{D.5 Joint estimation of $(t, D, \sigma_0)$}

A common objection is that the main results assume known $D$ and $\sigma_0$.
To address this, we distinguish two mathematically different settings:

\begin{enumerate}
    \item \textbf{Identifiable joint estimation (data provide sufficient information).}
    A finite frequentist joint CRLB $\mathrm{Var}(\hat t)\ge (\mathbf{J}^{-1})_{tt}$ exists only if the parameters are identifiable from the observation model, i.e. the Fisher information matrix $\mathbf{J}$ is full rank (or, more generally, $(\mathbf{J}^{-1})_{tt}$ is finite after restricting to the identifiable subspace).

    \item \textbf{Non-identifiable single-time setting (variance-only degeneracy).}
    If the likelihood depends on the parameters only through a single scalar combination (e.g. an effective variance), then $\mathbf{J}$ from the data is rank-deficient and the frequentist joint CRLB for $t$ is not finite without adding additional information (e.g.\ multi-time data, independent observables, or calibration constraints).
\end{enumerate}

For the Gaussian diffusion observation model with additive width floor,
\[
\mathbf{x} \sim \mathcal{N}\!\left(0,\, s\,\mathbf{I}_d\right),
\qquad
s(t)=2Dt+\sigma_0^2,
\]
a \emph{single-time} snapshot enters the likelihood only through the scalar $s$; thus the observation Fisher information in $(t,D,\sigma_0^2)$ is rank-1 and joint estimation is non-identifiable from that snapshot alone.
Accordingly, any finite ``joint'' bound in this single-time case necessarily encodes \emph{additional information} beyond the snapshot itself (for example, external calibration of $D$ and/or $\sigma_0^2),$ and is best interpreted as a posterior/Bayesian CRLB rather than a pure data-driven frequentist CRLB.

In contrast, with \emph{identifying information} (e.g.\ observations at multiple times, or additional independent measurements that break the degeneracy), the joint Fisher information matrix becomes full rank and a finite frequentist joint CRLB exists.
In that identifiable regime, nuisance couplings (off-diagonal terms) can inflate $\mathrm{Var}(\hat t)$ by a multiplicative factor, but they do not change the fundamental scaling exponents with $N_{\mathrm{eff}}$ and $\Phi$ that are set by the information content of the observation model.

Formally, we consider the joint Fisher information matrix
\[
\mathbf{J} =
\begin{pmatrix}
J_{tt} & J_{tD} & J_{t\sigma_0} \\
J_{Dt} & J_{DD} & J_{D\sigma_0} \\
J_{\sigma_0 t} & J_{\sigma_0 D} & J_{\sigma_0\sigma_0}
\end{pmatrix},
\]
where in practice $J_{t\sigma_0}$ denotes the coupling to the width-floor parameter (often represented as $\sigma_0^2$ in the Gaussian-variance parametrization).
When $\mathbf{J}$ is full rank, inverting $\mathbf{J}$ yields the joint CRLB
\[
\mathrm{Var}(\hat{t}) \ge (\mathbf{J}^{-1})_{tt}.
\]

\begin{figure}[h!]
\centering
\includegraphics[width=0.75\textwidth]{../figs/fig_D4_joint_estimation_penalty.pdf}
\caption{Penalty due to joint estimation.
The figure distinguishes an identifiable multi-time joint-CRLB regime (pure frequentist Fisher inversion) from a single-time, variance-degenerate regime where a finite bound requires additional calibration information (e.g.\ priors or external constraints).
In the identifiable regime, nuisance couplings can inflate the variance by a multiplicative factor but preserve the scaling exponents with $N_{\mathrm{eff}}$ and $\Phi$ implied by the Fisher-information structure.
In the single-time degenerate regime, the effective penalty can become calibration-limited.}
\label{fig:D4}
\end{figure}

Crucially, while nuisance parameters reduce available information through off-diagonal couplings, they cannot alter the asymptotic dependence of temporal distinguishability on sample number or photon flux \emph{in identifiable regimes where a frequentist joint CRLB is well-defined}.
When identifiability fails (e.g.\ single-time variance-only data), additional assumptions or measurements are required before any meaningful joint bound can be stated.

---

\subsection{D.6 Realistic single-particle tracking case study}

We consider a representative experimental scenario from the literature:
GFP diffusion in the cytoplasm with
\[
D \approx 10~\mu\mathrm{m}^2/\mathrm{s}, \quad
\sigma_0 \approx 250~\mathrm{nm}, \quad
\Phi \sim 10^4~\mathrm{s}^{-1}.
\]

Substituting these values yields a minimal resolvable time scale on the order of
\[
\Delta t_{\min} \sim 0.5\text{--}1~\mathrm{ms},
\]
consistent with reported experimental limitations in high-speed single-particle tracking.

For subdiffusive environments ($\alpha < 1$), the simulations confirm a rapid degradation of temporal distinguishability, quantitatively explaining the empirical difficulty of time inference in crowded intracellular media.

---

\subsection{D.7 Robustness and sensitivity analysis}

We tested robustness with respect to:
\begin{itemize}
    \item estimator choice (maximum likelihood versus moment-based),
    \item moderate correlations between samples,
    \item variation of the prefactor $\kappa$ within physically realistic bounds.
\end{itemize}

In all cases, only numerical prefactors changed; the scaling exponents and no-go character of the limits remained invariant.

---

\subsection{D.8 Summary of numerical evidence}

The numerical results demonstrate that:
\begin{enumerate}
    \item The CRLB-derived time distinguishability limit is tight and achievable.
    \item Photon-limited scaling $\Delta t_{\min} \propto \Phi^{-1/3}$ is robust and nontrivial.
    \item Joint estimation and estimator choice cannot circumvent the fundamental bounds (in identifiable regimes; in non-identifiable settings additional calibration/observables are required to state finite joint bounds).
\end{enumerate}

Together, these results establish the temporal distinguishability limit as a genuine operational constraint arising from information geometry, rather than a methodological artifact.

% ============================================================
