% ============================================================
% 06_anomalous_diffusion.tex
% ============================================================

\section{Anomalous Diffusion and Time Distinguishability}
\label{sec:anomalous_diffusion}

Many physical, chemical, and biological systems exhibit deviations from normal diffusion.
These are commonly described by a generalized mean-square displacement (MSD)
\begin{equation}
\langle r^2(t) \rangle = 2d\,D_\alpha\, t^\alpha,
\label{eq:anomalous_msd}
\end{equation}
where $0<\alpha<2$ is the anomalous exponent and $D_\alpha$ is the generalized diffusion
coefficient with units $[D_\alpha]=\mathrm{length}^2/\mathrm{time}^\alpha$.

In this section, we derive operational limits on temporal distinguishability for such
processes. Throughout, we assume self-similar propagators with well-defined MSD; the
Gaussian case is treated analytically, while more general cases are discussed
separately.

\subsection{MSD-Based Lower Bound (Interpretive Criterion)}

Before turning to information-theoretic bounds, it is instructive to consider a simple
resolution-based lower bound. For two close hypotheses $t$ and $t+\Delta t$, a necessary
(but not sufficient) requirement for distinguishability is that the expected change in
the MSD exceeds the effective spatial resolution scale $\Delta x^2$,
\begin{equation}
\Delta \langle r^2 \rangle
=
\frac{d}{dt}\langle r^2(t)\rangle\, \Delta t
\ge \Delta x^2.
\end{equation}

Using~\eqref{eq:anomalous_msd}, this yields
\begin{equation}
2d\,\alpha\,D_\alpha\, t^{\alpha-1}\,\Delta t
\ge \Delta x^2.
\end{equation}

Solving for $\Delta t$ gives the heuristic lower bound
\begin{equation}
\Delta t_{\mathrm{MSD}}(t)
=
\frac{\Delta x^2}{2d\,\alpha\,D_\alpha}\, t^{1-\alpha}.
\label{eq:anomalous_resolution}
\end{equation}

This criterion does not account for finite statistics, photon noise, or optimal
inference and therefore cannot be interpreted as a fundamental limit. It serves
solely as an interpretive baseline.

\subsection{Regime Structure (Interpretation)}

Equation~\eqref{eq:anomalous_resolution} reveals three qualitatively distinct behaviors.

\paragraph{Subdiffusion ($0<\alpha<1$).}

\[
\Delta t_{\mathrm{MSD}}(t) \propto t^{1-\alpha},
\]
which increases monotonically with time. Temporal distinguishability degrades
progressively.

\paragraph{Normal diffusion ($\alpha=1$).}

The bound reduces to a constant,
\[
\Delta t_{\mathrm{MSD}} = \frac{\Delta x^2}{2dD},
\]
recovering the classical resolution-limited diffusion time as a special case.

\paragraph{Superdiffusion ($1<\alpha<2$).}

\[
\Delta t_{\mathrm{MSD}}(t) \propto t^{1-\alpha},
\]
which decreases with time as spatial separation accelerates.

\subsection{Photon-Limited Extension}

We now return to the information-theoretic framework of
Section~\ref{sec:photon_limited}. For anomalous diffusion with a Gaussian propagator,
the observed variance satisfies
\begin{equation}
\sigma_{\mathrm{obs}}^2(t)
=
2d\,D_\alpha\, t^\alpha + \sigma_0^2.
\end{equation}

As before, photon counting obeys Poisson statistics with
$N_\gamma=\Phi\,\Delta t$, and the variance of the variance estimator satisfies
\begin{equation}
\mathrm{Var}(\widehat{\sigma^2})
=
\kappa \frac{\sigma_{\mathrm{obs}}^4}{N_\gamma},
\end{equation}
with $\kappa=2$ under ideal Gaussian conditions.

Propagating this uncertainty to time and imposing the self-consistency condition
$\Delta t_{\min}^2=\mathrm{Var}(\hat t)$ yields
\begin{equation}
\Delta t_{\min}^3
=
\frac{\kappa}{4d^2\alpha^2 D_\alpha^2 \Phi}
\left(2d\,D_\alpha\, t^\alpha + \sigma_0^2\right)^2
t^{2(1-\alpha)}.
\label{eq:anomalous_photon}
\end{equation}

This expression is the anomalous-diffusion analogue of the cubic fixed-point relation
derived for normal diffusion in Eq.~\eqref{eq:self_consistent}.

\subsection{Asymptotic Scaling Laws}

\paragraph{PSF-Dominated Regime ($\sigma_0^2 \gg 2dD_\alpha t^\alpha$).}

\begin{equation}
\Delta t_{\min}(t)
\propto
\frac{t^{\frac{2}{3}(1-\alpha)}}{\Phi^{1/3}}.
\end{equation}

\paragraph{Anomalous-Diffusion-Dominated Regime ($2dD_\alpha t^\alpha \gg \sigma_0^2$).}

\begin{equation}
\Delta t_{\min}(t)
\propto
\frac{t^{\frac{2-\alpha}{3}}}{\Phi^{1/3}}.
\end{equation}

In both regimes the characteristic photon-limited exponent $\Phi^{-1/3}$ is preserved,
demonstrating that the cubic self-consistency structure is universal across normal and
anomalous diffusion within the stated model class.

\subsection{Information-Theoretic Interpretation}

The anomalous exponent $\alpha$ controls how rapidly spatial uncertainty generates
information about elapsed time. Subdiffusion ($\alpha<1$) creates an information
bottleneck, while superdiffusion ($\alpha>1$) enhances temporal distinguishability.
The dependence on $\alpha$ modifies the temporal scaling but does not alter the
photon-limited exponent, which is fixed by the observational channel.

\subsection{Limitations}

The derivations above assume:
\begin{itemize}
\item self-similar propagators with well-defined MSD,
\item sufficiently weak temporal correlations to permit effective Fisher-information
      scaling,
\item asymptotically efficient estimators operating within a fixed observation model.
\end{itemize}

Strongly non-Gaussian or ageing processes (e.g., continuous-time random walks with
diverging waiting-time moments or Lévy flights) generally violate these assumptions.
Such cases require explicit numerical or nonparametric information-theoretic analysis
and are addressed separately in Appendix~B.
