% ============================================================
% 02_problem_setup.tex
% ============================================================

\section{Operational Problem Setup: Time as Distinguishability}
\label{sec:problem_setup}

In this section we formalize the notion of \emph{time distinguishability} as an
operational inference problem. The formulation is deliberately agnostic to any
ontological status of time and relies solely on observable spatial data and
decision-theoretic criteria.

\subsection{Time as an Inferred Parameter}

Let $x \in \mathbb{R}^d$ denote a spatial random variable generated by a stochastic
process at an unknown elapsed time $t \ge 0$. Observations consist of samples
\[
\{x_i\}_{i=1}^{N_{\mathrm{eff}}} \sim p(x \mid t, \vartheta),
\]
where $\vartheta$ denotes additional model parameters (e.g., diffusion coefficient,
noise floor) and $N_{\mathrm{eff}}$ is the effective sample size accounting for
correlations (Section~\ref{sec:assumptions_scope}).

Time $t$ is treated as an unknown continuous parameter to be inferred from spatial
statistics alone. No external clock signal or time-stamping is assumed.

\subsection{Two Complementary Notions of Distinguishability}

We adopt two complementary, but locally equivalent, operational notions of temporal
distinguishability.

\paragraph{Estimation-based distinguishability.}
Time distinguishability is defined via the minimal achievable uncertainty of any
unbiased estimator $\hat t$ constructed from the data. Given a target confidence
factor $z>0$ (e.g., a Gaussian $z$-score corresponding to a chosen confidence level),
we define
\begin{equation}
\Delta t_{\min}^{\mathrm{est}}(t; z)
:= z \sqrt{\mathrm{Var}(\hat t)} ,
\end{equation}
where $\mathrm{Var}(\hat t)$ is lower bounded by the Cramér--Rao inequality.
The numerical value of $z$ fixes a confidence convention but does not affect the
scaling behavior of $\Delta t_{\min}$.

\paragraph{Hypothesis-testing distinguishability.}
Alternatively, consider the binary hypothesis test
\[
H_0: t \quad \text{vs.} \quad H_1: t + \Delta t .
\]
Temporal distinguishability is quantified by the minimal $\Delta t$ such that the
two hypotheses can be discriminated with prescribed error probabilities. We define
\begin{equation}
\Delta t_{\min}^{\mathrm{KL}}(t; p^\ast)
:= \inf \left\{ \Delta t > 0 \;\middle|\;
D_{\mathrm{KL}}\!\left(p(x\mid t) \,\|\, p(x\mid t+\Delta t)\right)
\ge C(p^\ast) \right\},
\end{equation}
where $p^\ast$ denotes a target error level and $C(p^\ast)$ is a fixed constant
determined by standard bounds in statistical decision theory. As in the estimation
case, $C(p^\ast)$ affects only numerical prefactors, not asymptotic scalings.

\subsection{Local Equivalence Principle}

For regular parametric families satisfying standard smoothness and identifiability
conditions, the two notions above are locally equivalent. Specifically, for small
$\Delta t$ the KL divergence admits the expansion
\[
D_{\mathrm{KL}}\!\left(p(x\mid t) \,\|\, p(x\mid t+\Delta t)\right)
= \tfrac{1}{2} I(t)\,(\Delta t)^2 + o\!\left((\Delta t)^2\right),
\]
where $I(t)$ is the Fisher information with respect to $t$.
As a consequence, hypothesis-testing distinguishability and estimator variance are
governed by the same information-theoretic quantity up to fixed numerical constants.

This equivalence ensures that $\Delta t_{\min}$ is not an arbitrary construct but
a well-defined operational quantity, independent of the specific inference strategy
within the local regime.

\subsection{Resources and Constraints}

Temporal distinguishability depends on available resources:
\begin{itemize}
  \item \textbf{Statistical resources}: effective sample size $N_{\mathrm{eff}}$.
  \item \textbf{Spatial resolution}: observation noise floor $\sigma_0$.
  \item \textbf{Photon budget}: total detected photons $N_\gamma = \Phi \Delta t$
        in optical measurements.
\end{itemize}

All results in subsequent sections explicitly display their dependence on these
resources and identify regimes in which particular constraints dominate.

\subsection{What Is \emph{Not} Assumed}

We emphasize that the following are \emph{not} assumed:
\begin{itemize}
  \item A fundamental minimum or quantization of time.
  \item Deterministic trajectories or clock-like internal dynamics.
  \item Infinite data, infinite resolution, or noiseless measurements.
\end{itemize}

Temporal distinguishability emerges solely from the statistical structure of spatial
observations and the constraints of inference.

\subsection{Interpretational Remark}

Throughout this work, ``time resolution'' refers to an \emph{operational limit on
inference}, not to a modification of underlying dynamics. The same physical process
may admit arbitrarily fine temporal parametrization in principle, while remaining
experimentally indistinguishable below $\Delta t_{\min}$. This distinction underlies
the information-theoretic no-go statements established later.
