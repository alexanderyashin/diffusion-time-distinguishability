% ============================================================
% 04_kl_equivalence.tex
% ============================================================

\section{Hypothesis Testing View: KL Divergence and Local Equivalence to Fisher Information}
\label{sec:kl_equivalence}

In this section we establish the precise relationship between estimation-based and
hypothesis-testing-based notions of temporal distinguishability. We show that, under
standard regularity conditions, the Kullback--Leibler (KL) divergence between
distributions at times $t$ and $t+\Delta t$ is locally governed by the same Fisher
information that determines the Cramér--Rao bound.

\subsection{KL Divergence Between Nearby Times}

Consider two distributions from a regular parametric family
$p(x \mid t)$ and $p(x \mid t+\Delta t)$. The KL divergence is defined as
\begin{equation}
D_{\mathrm{KL}}\!\left(
p(x \mid t)\,\|\,p(x \mid t+\Delta t)
\right)
=
\mathbb{E}_{t}
\left[
\log \frac{p(X \mid t)}{p(X \mid t+\Delta t)}
\right].
\end{equation}

Assume that $p(x\mid t)$ is twice continuously differentiable in $t$ and that
differentiation can be interchanged with integration. A Taylor expansion of
$\log p(x \mid t+\Delta t)$ around $t$ yields
\begin{equation}
\log p(x \mid t+\Delta t)
=
\log p(x \mid t)
+
\Delta t \, \partial_t \log p(x \mid t)
+
\frac{(\Delta t)^2}{2} \, \partial_t^2 \log p(x \mid t)
+
R_3(x,\Delta t),
\end{equation}
where the remainder satisfies
$|R_3(x,\Delta t)| \le C |\Delta t|^3$ uniformly in a neighborhood of $t$ under
standard regularity assumptions (bounded third derivatives and dominated
convergence).

Substituting into the definition of $D_{\mathrm{KL}}$ and using
$\mathbb{E}_t[\partial_t \log p(X\mid t)] = 0$, we obtain
\begin{equation}
D_{\mathrm{KL}}\!\left(
p(x \mid t)\,\|\,p(x \mid t+\Delta t)
\right)
=
\frac{(\Delta t)^2}{2}
\mathbb{E}_t\!\left[
\left(\partial_t \log p(X \mid t)\right)^2
\right]
+ O\!\left((\Delta t)^3\right).
\end{equation}

The expectation in the leading term is precisely the Fisher information $I_1(t)$.
Hence,
\begin{equation}
D_{\mathrm{KL}}\!\left(
p(x \mid t)\,\|\,p(x \mid t+\Delta t)
\right)
=
\frac{1}{2} I_1(t) (\Delta t)^2
+ O\!\left((\Delta t)^3\right).
\end{equation}

\subsection{Multiple Observations}

For $N_{\mathrm{eff}}$ independent or effectively independent observations, KL
divergence is additive:
\begin{equation}
D_{\mathrm{KL}}^{(\mathrm{total})}
=
N_{\mathrm{eff}}
D_{\mathrm{KL}}^{(1)} .
\end{equation}

Therefore,
\begin{equation}
D_{\mathrm{KL}}^{(\mathrm{total})}(t,\Delta t)
=
\frac{1}{2}
I_{\mathrm{eff}}(t)
(\Delta t)^2
+ O\!\left(N_{\mathrm{eff}}(\Delta t)^3\right).
\end{equation}

\subsection{Operational Thresholds and Error Probabilities}

To relate KL divergence to hypothesis-testing performance, consider a binary test
between hypotheses $H_0: t$ and $H_1: t+\Delta t$. Standard inequalities in
information theory relate KL divergence to achievable error probabilities.

In particular, the Bretagnolle--Huber inequality implies
\begin{equation}
P_e \ge \frac{1}{4} \exp\!\left(-D_{\mathrm{KL}}^{(\mathrm{total})}\right),
\end{equation}
where $P_e$ is the minimal achievable total error probability.

Fixing a target error level $P_e \le p^\ast$, it suffices to require
\begin{equation}
D_{\mathrm{KL}}^{(\mathrm{total})}
\ge
C(p^\ast)
:=
\log\!\left(\frac{1}{4 p^\ast}\right).
\end{equation}

Alternative choices based on Pinsker, Chernoff, or symmetric KL bounds lead to
equivalent criteria that differ only by fixed numerical factors and do not modify
any scaling laws.

\subsection{Equivalence of Estimation and Testing Criteria}

Combining the local KL expansion with the threshold above yields
\begin{equation}
\Delta t_{\min}^{\mathrm{KL}}(t; p^\ast)
=
\sqrt{
\frac{2 C(p^\ast)}{I_{\mathrm{eff}}(t)}
}
\left[1 + O(\Delta t)\right].
\end{equation}

Up to the explicit constant $\sqrt{2C(p^\ast)}$, this coincides with the
estimation-based distinguishability $\Delta t_{\min}^{\mathrm{CRLB}}$ derived in
Section~\ref{sec:normal_diffusion_crlb}. The two notions are therefore locally
equivalent and define the same operational temporal resolution.

\subsection{Interpretational Remark}

The equivalence established here ensures that temporal distinguishability is not an
artifact of a particular inference paradigm. Whether time is inferred via parameter
estimation or discriminated via hypothesis testing, the fundamental limitation is set
by the same Fisher information of the spatial distribution.
