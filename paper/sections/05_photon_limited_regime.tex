% ============================================================
% 05_photon_limited_regime.tex
% ============================================================

\section{Photon-Limited Regime and Self-Consistent Temporal Resolution}
\label{sec:photon_limited}

In many experimental settings—single-particle tracking (SPT), fluorescence correlation
spectroscopy (FCS), and localization microscopy—the dominant limitation is not the number
of particles but the finite photon flux. In this regime, the effective statistical
information available for inference depends explicitly on the temporal resolution
$\Delta t$ being probed. As a result, temporal distinguishability must be determined
self-consistently.

\subsection{Photon Statistics and Observation Noise}

We assume that during an observation window of duration $\Delta t$, the total number of
detected photons follows Poisson statistics,
\begin{equation}
N_\gamma \sim \mathrm{Poisson}(\Phi\, \Delta t),
\end{equation}
where $\Phi$ is the photon flux (photons per unit time). The quantity $\Delta t$ here
denotes the acquisition or exposure duration and should be distinguished from the
minimal distinguishable time increment $\Delta t_{\min}$ derived below.

The observed spatial variance is
\begin{equation}
\sigma_{\mathrm{obs}}^2(t)
=
2Dt + \sigma_0^2,
\end{equation}
where $\sigma_0^2$ denotes the static point-spread-function (PSF) floor.

\subsection{Variance of the Variance Estimator}

For a Gaussian spatial profile estimated from $N_\gamma$ detected photons, the variance
of the maximum-likelihood estimator of the variance satisfies (see Appendix~B for a
derivation)
\begin{equation}
\mathrm{Var}(\widehat{\sigma^2})
=
\kappa \frac{\sigma_{\mathrm{obs}}^4}{N_\gamma},
\label{eq:var_sigma_photon}
\end{equation}
where $\kappa = 2$ for an ideal Gaussian PSF and asymptotically efficient estimators in
the large-photon limit. This coefficient is fixed by the statistical model and is not a
free parameter.

\subsection{Propagation of Uncertainty to Time}

Using $\sigma_{\mathrm{obs}}^2 = 2Dt + \sigma_0^2$, standard error propagation yields
\begin{equation}
\mathrm{Var}(\hat t)
=
\frac{\mathrm{Var}(\widehat{\sigma^2})}{(2D)^2}
=
\frac{\kappa}{4D^2}
\frac{(2Dt + \sigma_0^2)^2}{N_\gamma}.
\end{equation}

Substituting $N_\gamma = \Phi\, \Delta t$ gives
\begin{equation}
\mathrm{Var}(\hat t)
=
\frac{\kappa}{4D^2}
\frac{(2Dt + \sigma_0^2)^2}{\Phi\, \Delta t}.
\end{equation}

\subsection{Self-Consistent Temporal Resolution}

The minimal resolvable time interval is defined operationally by requiring that the
uncertainty of the inferred time matches the temporal resolution being tested,
\begin{equation}
\Delta t_{\min}^2
=
\mathrm{Var}(\hat t),
\end{equation}
which is consistent with the estimator-based distinguishability criterion used in
previous sections up to fixed numerical confidence factors.

This yields the self-consistency (fixed-point) equation
\begin{equation}
\Delta t_{\min}^3
=
\frac{\kappa}{4D^2 \Phi}
(2Dt + \sigma_0^2)^2.
\label{eq:self_consistent}
\end{equation}

The cubic structure of Eq.~\eqref{eq:self_consistent} reflects a closed information
loop: improving temporal resolution requires more photons, while acquiring photons
requires time.

\subsection{Two Asymptotic Regimes}

\paragraph{PSF-dominated regime ($\sigma_0^2 \gg 2Dt$).}

In this limit,
\begin{equation}
\Delta t_{\min}
=
\left(
\frac{\kappa\, \sigma_0^4}{4D^2 \Phi}
\right)^{1/3}.
\end{equation}

This yields the nontrivial scaling law
\[
\Delta t_{\min} \propto \Phi^{-1/3},
\]
which is fundamentally distinct from the standard $1/\sqrt{N}$ noise suppression and
cannot be obtained by dimensional analysis alone.

\paragraph{Diffusion-dominated regime ($2Dt \gg \sigma_0^2$).}

Here,
\begin{equation}
\Delta t_{\min}
=
\left(
\frac{\kappa\, t^2}{\Phi}
\right)^{1/3}
=
\kappa^{1/3}\,
\frac{t^{2/3}}{\Phi^{1/3}}.
\end{equation}

Temporal resolution degrades algebraically with absolute time, reflecting the loss of
distinguishability as diffusion broadens.

\subsection{Interpretation}

The cubic self-consistency relation~\eqref{eq:self_consistent} is the central result of
the photon-limited regime. The exponent $1/3$ is not heuristic but follows uniquely
from the combination of Poisson photon statistics and quadratic spatial uncertainty.
Any inference scheme operating within this observational channel and noise model must
obey the same scaling.

\subsection{Relation to Fisher Information}

The photon-limited bound can alternatively be derived from a Fisher-information
calculation in which the photon number $N_\gamma=\Phi\Delta t$ is treated as a
stochastic, time-dependent resource. Both approaches yield identical scaling and
consistent numerical prefactors under matching modeling assumptions, a fact confirmed
by numerical simulations in Appendix~D.
