% ============================================================
% 09_conclusion.tex
% ============================================================

\section{Conclusion}
\label{sec:conclusion}

We have presented a fully operational and information-theoretic framework
for defining minimal temporal distinguishability in diffusive systems.

The central result is simple to state and nontrivial in consequence:
\emph{time resolution is not a primitive quantity but an emergent property
of statistical distinguishability between probability distributions}.

\subsection*{Summary of Results}

\begin{enumerate}
\item Time was treated as an inferred parameter of spatial distributions,
      not as an externally given variable.
\item For normal diffusion, Fisher information yields a universal bound
      $\Delta t_{\min}(t) \sim t \sqrt{2/(dN)}$ under the stated assumptions.
\item Hypothesis testing via KL divergence leads to an equivalent local bound,
      establishing estimator-independent consistency.
\item In the photon-limited regime, a self-consistent analysis predicts
      a nontrivial scaling $\Delta t_{\min} \propto \Phi^{-1/3}$,
      distinct from standard $1/\sqrt{N}$ noise suppression.
\item For anomalous diffusion, qualitatively different temporal regimes
      emerge, governed by the exponent $\alpha$.
\end{enumerate}

All results are analytically derived, dimensionally consistent,
and experimentally falsifiable within controlled uncertainty bounds.

\subsection*{Conceptual Implications}

The framework clarifies a long-standing ambiguity in stochastic dynamics:
what it means to \emph{resolve time} in systems without intrinsic clocks.

Rather than postulating time as fundamental, the theory shows that
temporal resolution is constrained by:

\begin{itemize}
\item statistical sample size,
\item measurement noise,
\item the dynamical law governing state evolution.
\end{itemize}

This perspective unifies estimation theory, hypothesis testing,
and physical diffusion under a single operational principle.

\subsection*{Experimental Outlook}

All predicted scaling laws are directly testable with existing techniques,
including single-particle tracking, fluorescence correlation spectroscopy,
and high-speed microscopy.

In particular, the $\Phi^{-1/3}$ photon-limited regime provides
a clear and decisive experimental target.

Systematic and statistically significant deviations from the predicted
exponents would falsify the framework for the corresponding regime.

\subsection*{Broader Context}

Although motivated by diffusion, the approach is general.
Any process described by a parameterized family of probability distributions
admits an analogous analysis.

The present work therefore establishes a template for studying
temporal distinguishability in a wide class of stochastic and
inference-driven systems.

% ============================================================
% ADDED: Quantum outlook (future work)
% ============================================================
\subsection*{Outlook: Quantum Limits}

The present analysis is entirely classical and relies on spatial
probability distributions derived from diffusive dynamics.
Nevertheless, the operational structure of temporal inference
suggests a natural extension to quantum metrological settings.

In regimes where quantum states of light or matter are employed,
photon statistics and quantum Fisher information may impose
additional bounds on temporal distinguishability.
Exploring such quantum-limited extensions lies beyond the scope
of the present work but represents a natural direction for future research.
% ============================================================

\subsection*{Final Remark}

No modification of physical laws is proposed.
No metaphysical assumptions are introduced.

The only claim is that time, when operationally defined,
is limited by information.

This limit is precise, universal within the stated model classes,
and measurable.
